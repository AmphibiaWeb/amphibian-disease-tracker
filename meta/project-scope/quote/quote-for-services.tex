%\documentclass{amsart}
%\usepackage{amsmath,amsfonts,amssymb,hyperref,graphicx,fullpage}
\documentclass{article}
\usepackage{hyperref, fullpage}
%\usepackage[none]{hyphenat}
\usepackage{fancyhdr}
\pagestyle{fancy}
\fancyhf{}
\lfoot{\small{Velociraptor Systems}}
\rfoot{\small{Anything not covered here covered by {\tt \href{http://goo.gl/KFevGx}{http://goo.gl/KFevGx}}}}
%\renewcommand{\footrulewidth}{0.4pt}

\fancypagestyle{plain}{}
\begin{document}

\title{Amphibian Disease tracking portal}

\author{
  \centering
  Philip Kahn\\
  & 1314 Hopkins St.\\
  Berkeley, CA 94702\\
  {\tt \href{mailto:support@velociraptorsystems.com}{support@velociraptorsystems.com}} \\
  (510) 859-3142 \\
  }
%\author{Philip Kahn\\1314 Hopkins St.\\Berkeley CA 94702\\ }

\maketitle


\section{Scope}

\begin{itemize}
  \item Integration of data from \textit{Bd}-Maps
  \item Integrate Bd and Bsal data from Vredenburg and participating labs
  \item Evaluate and potentially use some code from the Ranavirus tracker by the Eco Health Alliance
  \item Mapping as a first tier citizen - directly queryable and sortable maps
  \item Utilize taxonomy from AmphibiaWeb database
\end{itemize}

\section{Project Date}

The project is estimated to begin in  August 2015. No target release date has yet been set.

\section{Quote}

This quote is for $\approx$ 5 weeks of work.

\section{Project Contact}

The contact for this project is\\

\noindent Michelle Koo\\
AmphibiaWeb, associate director\\
{\tt mkoo@berkeley.edu}\\



\noindent \textbf{Initial Quote:} \$5000


\end{document}
